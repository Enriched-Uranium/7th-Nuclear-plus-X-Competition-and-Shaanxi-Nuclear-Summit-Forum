\documentclass{ctexbeamer}
\usetheme{metropolis}
\usepackage[no-math]{fontspec}
\usepackage[T1]{fontenc}
\usepackage{mathpazo}
\setCJKsansfont{SimHei}
\setsansfont{Palatino}
\setmonofont{Consolas}
\usepackage{graphicx}
\usepackage{xcolor}
\usepackage{amsmath}
\usepackage{bm}
% 算法环境
\usepackage{algorithm}  
\usepackage{algorithmicx}  
\usepackage{algpseudocode}
\usepackage{fontawesome5}
\usefonttheme{professionalfonts}
\title{第7届“核+ X”决赛\&陕核高峰论坛纪要}
\date{\today}
\author{一小块浓缩铀 \quad \faRadiation* \quad \href{https://github.com/Enriched-Uranium/Nuclear-Reactor-Physics}{\faGithub}}
\institute{西安交通大学 \quad 核科学与技术学院}
\begin{document}
% xelatex --synctex=1 CI.tex
\maketitle

\begin{frame}[standout]
    \huge 第7届“核+ X”决赛
\end{frame}

\begin{frame}{第7届“核+ X”决赛}
    \begin{enumerate}
        \item 上海交通大学:方舟反应堆制作指南
        \begin{enumerate}
            \item 核聚变:氘氚反应;
            \item 托卡马克磁约束装置;
            \item 新世界的钥匙,东方超环EAST实现1056秒运行,环流器2号HL-2M实现1 MA放电;
            \item 工业革命是能源转换的革命。
        \end{enumerate}
        \item 哈尔滨工程大学:辐射不育技术
        \begin{enumerate}
            \item ${}^{60}{\rm Co}$和${}^{137}{\rm Cs}$衰变放出$\gamma$射线辐射不育;
            \item 生殖细胞对辐射的敏感程度比体细胞更高;
            \item 核技术昆虫防治和其他应用。
        \end{enumerate}
        \item 南华大学:跟着辐宝去造福
        \begin{enumerate}
            \item 低能电子束(10 MeV)灭活耐低温病毒,阻止冷链食品传播;
            \item 不穿透外包装且无残留,效率高;
            \item 辐照育种技术,提高产量和营养价值。
        \end{enumerate}
    \end{enumerate}
\end{frame}

\begin{frame}{第7届“核+ X”决赛}
    \begin{enumerate}\setcounter{enumi}{3}
        \item 南京航空航天大学:$\gamma$射线高度计
        \begin{enumerate}
            \item 返回舱着陆前紧急制动的高度判断(1 m左右);
            \item $\gamma$源和闪烁体接受器,毫厘级精度,抗干扰性强,实现有限燃料的有效制动;
            \item 核技术在航天领域的应用。
        \end{enumerate}
        \item 核工业学院:“核”心跳动
        \begin{enumerate}
            \item 核电池心脏起搏器,终生不用更换电池;
            \item 放射性同位素衰变放出载能粒子,通过热电材料转换(塞贝克效应);
            \item 能量密度高,体积小,寿命长,不会造成辐射损伤。
        \end{enumerate}
        \item 苏州大学:平地惊“镭”
        \begin{enumerate}
            \item 镭产生$\alpha$衰变对肿瘤细胞杀伤效果好;
            \item 镭和钙同主族,对骨癌治疗靶向性强。
        \end{enumerate}
    \end{enumerate}
\end{frame}

\begin{frame}{第7届“核+ X”决赛}
    \begin{enumerate}\setcounter{enumi}{6}
        \item 西安交通大学:蚊核友
        \begin{enumerate}
            \item 辐照灭蚊技术,射线破坏雄蚊的生殖细胞,小剂量辐照致使完全不育,但雄蚊求偶能力有一定下降;
            \item 消灭携带致命病毒的蚊子,防治传染病;
            \item 亚不育剂量辐照使得染色体易位,辐照雄蚊求偶能力与正常雄蚊相当。
        \end{enumerate}
        \item 核工业学院:遥相应核,月来月好
        \begin{enumerate}
            \item 放射性同位素电池(千瓦级)月球核反应堆电源(百千瓦级)供电;
            \item 屏蔽层,深埋月壤等措施削弱辐射。
        \end{enumerate}
    \end{enumerate}
\end{frame}

\begin{frame}[standout]
    \huge 陕核高峰论坛
\end{frame}

\begin{frame}{潘德炉院士:世界百年变局对我国科技根基的思考}
    \begin{enumerate}
        \item 中国的崛起挑战美国霸权
        \begin{enumerate}
            \item 邓小平:“韬光养晦、有所作为”“科技是第一生产力”。
            \item 新型大国关系:相互尊重、和平共处、合作共赢。
            \item 美国将中国视为“长期、头号战略竞争对手”。
            \item 拓展利益看海上权利和海洋安全。
        \end{enumerate}
        \item 对我国科技根基的思考
        \begin{enumerate}
            \item 要有破釜沉舟的魄力和决心研发高端芯片。
            \item 加快通导遥一体化星链网络建设。
            \item 发展自主可控大数据核心算法。
            \item 自主可控的操作系统是软件之本。
            \item 人才是国家科技竞争力核心根基。
        \end{enumerate}
    \end{enumerate}
\end{frame}

\begin{frame}{邱爱慈院士:脉冲功率技术:能源开发的新手段}
    \begin{enumerate}
        \item 脉冲功率技术概述
        \begin{enumerate}
            \item 初级储能,脉冲功率系统,负载。
            \item 储能、开关、绝缘、传输线、负载和测量技术。
            \item 核爆模拟、高新武器、核聚变、能源开发。
        \end{enumerate}
        \item 聚变能源技术发展现状
        \begin{enumerate}
            \item 劳逊判据(氘氘反应)$T>100 {\,\rm keV}$,$n\tau > 10^{16}{\,\rm {cm}^{-3}\cdot s}$。
            \item 磁约束(托卡马克、仿星器)、激光惯性约束(ICF)。
            \item 国际热核聚变堆ITER,东方超环EAST,美国国家点火工程(NIF)。
            \item Z箍缩聚变科学技术,产生高温、高压、高密、高速和强辐射环境。
        \end{enumerate}
        \item Z箍缩聚变裂变反应堆
        \begin{enumerate}
            \item 聚变-裂变混合能源是未来清洁能源的重要方向。
            \item Z箍缩驱动器+局部体点火靶+深次临界能源堆。
            \item Cz-15装置,用于强辐射模拟、材料科学、天体物理等。
        \end{enumerate}
    \end{enumerate}
\end{frame}

\begin{frame}{邱爱慈院士:脉冲功率技术:能源开发的新手段}
    \begin{enumerate}\setcounter{enumi}{3}
        \item 能源开发的新手段
        \begin{enumerate}
            \item 国家能源安全形势紧迫,西部富油煤潜力巨大,煤炭生产安全形势严峻。
            \item 可控冲击波技术:大电流在金属丝局部迅速释放能量。
            \item 冲击波致裂砂岩效应实验、常规油水井增注、煤矿钻孔瓦斯治理。
        \end{enumerate}
        \item 前景和展望
        \begin{enumerate}
            \item 保障国家油气和煤炭供给。
            \item 解决高温高压环境中应用的难题,推进页岩油、富油煤的开发。
        \end{enumerate}
    \end{enumerate}
\end{frame}

\begin{frame}{于俊崇院士:实现“双碳”——核能的机遇与挑战}
    \begin{enumerate}
        \item 实现“双碳”的意义和难度
        \begin{enumerate}
            \item 拯救地球气候,改善人类居住环境,推动国家高质量发展,建立人与自然和谐共生的生态文明,发展清洁能源实现能源自给。
            \item 时间紧、任务重、难度大,众多清洁能源技术需要开发,50\%以上的技术尚不成熟。
        \end{enumerate}
        \item 核能的机遇和挑战
        \begin{enumerate}
            \item 机遇:能量密度高,技术成熟,供热稳定,减排效果显著,党和国家寄予期望和鼓励。
            \item 挑战:储能技术(钠电池、钙钛矿电池),高新电网技术(柔性、分布式智能电网),光伏上网电价低、转换效率高,风光电制造规模大、速度快,煤的清洁利用技术。
        \end{enumerate}
    \end{enumerate}
\end{frame}

\begin{frame}{于俊崇院士:核能的机遇与挑战}
    \begin{enumerate}\setcounter{enumi}{2}
        \item 核能必须与时俱进
        \begin{enumerate}
            \item 以新技术改造“二代+”核电机组。
            \item 积极有序发展三代核电。
            \item 开发适用不同场合的微核反应堆。
            \item 加大核能利用产业链中薄弱环节技术研发。
            \item 加大研发先进核能技术及应用。
        \end{enumerate}
        \item 小结
        \begin{enumerate}
            \item 实现“双碳”是一场深刻变革,意义重大。
            \item 实现“双碳”过程中需要每一种能源做贡献。
            \item 核能是人类不可或缺的能源。
            \item 每个人都是节能减排的变革者,也是建设生态文明的变革对象。
        \end{enumerate}
    \end{enumerate}
\end{frame}

\begin{frame}{罗琦院士:新一代核能技术特征与发展方向}
    \begin{enumerate}
        \item 我国严峻的能源形势
        \begin{enumerate}
            \item 能源消费快速增长,供应需求矛盾突出。
            \item 对外依存度高(石油、天然气、铀),能源安全形势严峻。
            \item 能源结构偏煤炭,能源转型降碳难度大。
            \item 产业结构偏重工业,节能降耗难度高。
        \end{enumerate}
        \item 我国发展核能的迫切需求
        \begin{enumerate}
            \item “双碳”目标下,逐步满足我国能源发展需要。
            \item 国家能源安全需求,世界能源强国(美、俄、法)以核能作为战略优先选项。
            \item 核能创新引领急迫。
        \end{enumerate}
        \item 我国核能现状
        \begin{enumerate}
            \item 以压水堆为主的基本格局,发展热堆快堆二元协同。
            \item 核电发电量远低于世界平均水平。
            \item 四代堆与预期安全性、经济性存在差距,尚未形成重点发展堆型。
        \end{enumerate}
    \end{enumerate}
\end{frame}

\begin{frame}{罗琦院士:新一代核能技术特征与发展方向}
    \begin{enumerate}\setcounter{enumi}{3}
        \item 新一代核能主要目标
        \begin{enumerate}
            \item 需求牵引,技术前迎,回应关切。
            \item 安全性问题为核心(“固有安全”型),经济性(“经济高效”型)、环境影响(“环境友好”型)和铀资源问题(“资源节约”型)并重考虑。
        \end{enumerate}
        \item 新一代核能主要特征
        \begin{enumerate}
            \item 反应堆具有固有安全特性,不存在堆芯熔化风险,放射性自我包容。
            \item 经济高效,发电成本低,热效率和发电效率高。
            \item 环境友好,全域选址,零污染、零排放。
            \item 资源利用率大幅提升,先进乏燃料处理技术,燃料利用率高。
        \end{enumerate}
    \end{enumerate}
\end{frame}

\begin{frame}{罗琦院士:新一代核能技术特征与发展方向}
    \begin{enumerate}\setcounter{enumi}{5}
        \item 新一代核能主要堆型
        \begin{enumerate}
            \item 一体化快堆(池式钠冷快堆),蓄热能力巨大。
            \item 固有安全在线增殖自然反应堆(铅冷行波堆),$k_{\rm eff} = 1$保持稳定,燃料利用率提高(一次装料,燃至退役),环境影响下降。
            \item 改进型四代堆(钠冷快堆,超高温气冷堆、熔盐堆)。
        \end{enumerate}
    \end{enumerate}
\end{frame}

\end{document}